\documentclass[a4paper, usenatbib, 12pt]{article}
\usepackage{subfig}
\usepackage{float}
\usepackage{wrapfig}
\usepackage{graphicx}
\usepackage{amsmath}
\usepackage{amssymb}
\usepackage{booktabs}
\usepackage{cite}
\usepackage[top=3.5cm, bottom=2.5cm, left=3.5cm, right=3.5cm]{geometry}

%----------------------New commands -------------------

\newcommand{\tol}{TOL1214-277}
\newcommand{\lya}{Ly$\alpha$}
\newcommand{\hb}{H$\beta$}
\newcommand{\ha}{H$\alpha$}
\newcommand{\oiii}{[OIII]}
\newcommand{\oii}{[OII]}
\newcommand{\nii}{[NII]}
\newcommand{\esca}{erg cm$^{-2}$ s$^{-1}$ \AA$^{-1}$}
\newcommand{\esc}{erg cm$^{-2}$ s$^{-1}$}
\newcommand{\es}{erg s$^{-1}$}
\newcommand{\esa}{erg s$^{-1}$}
\newcommand{\kms}{km s$^{-1}$}
\newcommand{\apj}{ApJ}  
\newcommand{\jcap}{JCAP}  
\newcommand{\apjs}{ApJS}  
\newcommand{\apjl}{ApJL}  
\newcommand{\aj}{AJ}  
\newcommand{\mnras}{MNRAS}  
\newcommand{\mnrassub}{MNRAS accepted}  
\newcommand{\aap}{A\&A}  
\newcommand{\aaps}{A\&AS}  
\newcommand{\araa}{ARA\&A}  
\newcommand{\nat}{Nature}  
\newcommand{\physrep}{PhR}
\newcommand{\pasp}{PASP}    
\newcommand{\pasj}{PASJ}    
\def\simgt{\lower.5ex\hbox{\gtsima}}
\def\simlt{\lower.5ex\hbox{\ltsima}}
%------------------------------------------------------

\begin{document}
\pagestyle{empty}
\noindent
\textbf{Lyman-alpha emission reveals an unusual fast rotating dwarf galaxy}  
\\
\\
JEFR$^{1}$, MCRG$^1$, JNGC$^2$,
\\
\\
\scriptsize
{$^1$ Universided de los Andes
\\
$^2$ Arizona
\normalsize
\\
\\
\textbf{The Lyman-alpha emission line is a strong indicator of star formation.
Observations of this line are central to construct samples of the most
distant star forming galaxies, which in turn are useful for studies in
cosmology and galaxy evolution.  
The intepretation of Ly-$\alpha$ observations has to go together with
computational models for the radiative transfer of Lyman-alpha photons.
Recent theoretical work suggests that galaxy rotation has a measurable
impact on the shape of the Lyman-alpha line.  
Here we report that lyman-alpha observations of a blue dwarf compact galaxy (TOL
1214-277) in the local universe, which had previously evaded theoretical 
interpretation, are naturally explained by rotational effects.
We constrain the rotational velocity, viewing angle and total
neutral hydrogen mass from the Lyman-alpha observations.
These values are in broad agreement with other observational
constraints, although  they point towards a rather atypical physical
nature for the source.  
Our results present a new observational method to estimate the rotational
velocity of dwarf galaxies.
Considering the expected similarities between the local and the most
distant dwarf galaxies, we anticipate that the Lyman-alpha line could be
used to constraint the rotational state of the neutral gas in the
highest redshift galaxies to be detected with next generation infrared
spectroscopic facilities such as the James Webb Space Telescope.}



%\section{Introduction}

1. General paragraph about the Lyman alpha line.

2. General paragraph about modelling the Lyman alpha
line. Outflows.

3. Rotation and the expected features.
It has been shown that rotation also imprints an effect on the
Lyman-alpha morphology. 
The most important consequence of rotation is that spherical 
symmetry is broken.
The line morphology now depends on the viewing angle respect to the
rotation axis.  
For a line of sight perpendicular to the rotation axis the intensity and
the line center and the line width increase with rotational velocity. 
When the rotational velocity is close to the half-line width of the
static line, the line becomes single peaked as it is observed in
\tol, a unique feature that other theoretical models find
impossible to reproduce.



%\section{TOL1214-277}

4. The charachteristics of the dwarf galaxy of interest.

\tol was first observed by ... it is a compact 
dwarf galaxy and does not have old stars. 

The Ly$\alpha$ emission line was first observed in the TOL1214-277 
galaxy by \cite{Thuan97}. It has two main important features which 
make this a very uncommun LAE. First it shown a symmetric profile 
which is , Second the Ly$\alpha$ line is not shifted with respect to 
the H$\beta$ line.  Blue compact Dwarf Galaxy

	%\section{Results}

5. The results of the fit. 

Figure 1. shows the observational data for \tol with the overplot from
our best fit model from the full radiative transfer simulation. 
The parameters for the best fit are
$v_{max}=300$\kms, $\tau=1\times10^7$, $T=1.5\times 10^{4}$K and a
viewing angle $\theta<30$ degrees. 


Observed line + fit. 


%6. Comparison of these results against other observational data.
Assuming spherical symmetry and a homogeneous gas distribution we
estimate the total neutral hydrogen mass to be on the order of
$M\approx m_H\tau^3\sigma^{-3}n^{-2}$, where $m_H$ is the mass a
Hydrogen's atom, $\tau$  the optical depth, $\sigma$ is the cross
section at the line's center and $n$ is the number density of neutral
Hydrogen atoms. 
For this system we estimate that for average values of $n=1\times
10^3$ the total hydrogen mass is $M\times 10^{14}$M$_{\odot}$. 
However, blind HI
surveys have put an upper limit in the neutral hydrogen mass of ???


7. Implications for outflow+rotation in existing samples.

%8. Implications for very high-z dwarf galaxies. 


%=====



\bibliography{references}{}
\bibliographystyle{plain}

\newpage
\begin{table}
\begin{center}

\begin{tabular}{lc}
$\alpha$(2000)$^{a}$ & 12h17min17.1s\\
$\delta$(2000)$^{b}$ & -28d02m32s\\
$l$, $b$ (deg) & 294, 34\\
$m_V$ & 17.5\\
$v$(km s$^{-1}$) & 7795
\end{tabular}
\end{center}

\caption{Observational characteristics of TOL1214-277 \cite{Thuan97}\\
$^{a}$ Units of right ascension are hours, minutes and
  seconds.\\ $^{b}$ Units of declination are degrees, arcminutes and arcseconds.}
\end{table}

The metallicity is $\sim Z_{\odot}/24$ \cite{Izotov04} as derived from optical
spectroscopy. The observed flux for the Lyman alpha line is $\sim
8.1\times 10^{-14}$ erg cm$^{-2}$ s$^{-1}$ \AA$^{-1}$ \cite{Thuan97}
and a Equivalent Width of $70$\AA. 

Interpretation by \cite{mashesse03}.

For an homogeneous sphere the HI optical depth can be written
as $\tau = \sigma_0 n s$, where $\sigma_0$ is the optical depth at the
line's center, $n$ is the number density and $s$ is the sistem's
size. 
This can be rewritten in terms of the gas' temperature $T$ and column
density $N_{H}$as $\tau = 3.31 \times 10^{-14} (10^{4}\mathrm{K}/T)^{1/2}
(N_{H}/\mathrm{atoms\ cm}^{-2})$.  

This allows us to approximate the total hydrogen mass as
\begin{equation}
M_{H} = m_{H}  N_{H} D^{2} = 226\times \tau  \left(\frac{T}{10^4
  \mathrm{K}}\right)^{1/2}\left(\frac{D}{\mathrm{kpc}}\right)^2M_{\odot}
\end{equation}

On the other hand, we have an estimate for the dynamical mass from the
galaxy size $D$ and its rotational velocity $V$:

\begin{equation}
M_{T} = \frac{V^{2}D}{G} = 4.32\times10^{5}
\left(\frac{V}{\mathrm{km\ s}^{-1}}\right)^2\left(\frac{D}{\mathrm{kpc}}\right) M_{\odot}
\end{equation}
\end{document}

