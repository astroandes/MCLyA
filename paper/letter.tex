\documentclass[a4paper, 12pt]{article}
\usepackage{subfig}
\usepackage{float}
\usepackage{wrapfig}
\usepackage{graphicx}
\usepackage{amsmath}
\usepackage{amssymb}
\usepackage{booktabs}
\usepackage[top=3.5cm, bottom=2.5cm, left=3.5cm, right=3.5cm]{geometry}

%----------------------New commands -------------------

\newcommand{\tol}{TOL1214-277}
\newcommand{\hb}{H$\beta$}
\newcommand{\ha}{H$\alpha$}
\newcommand{\oiii}{[OIII]}
\newcommand{\oii}{[OII]}
\newcommand{\nii}{[NII]}
\newcommand{\esca}{erg cm$^{-2}$ s$^{-1}$ \AA$^{-1}$}
\newcommand{\esc}{erg cm$^{-2}$ s$^{-1}$}
\newcommand{\es}{erg s$^{-1}$}
\newcommand{\esa}{erg s$^{-1}$}
\newcommand{\kms}{km s$^{-1}$}
\def\simgt{\lower.5ex\hbox{\gtsima}}
\def\simlt{\lower.5ex\hbox{\ltsima}}
%------------------------------------------------------

\begin{document}
\pagestyle{empty}
\noindent
\textbf{Measurement of a dwarf galaxy rotational velocity from its
  Lyman-alpha line emission} 
\\
\\
JEFR$^{1}$, MCRG$^1$, JNGC$^1$,
\\
\\
\scriptsize
{$^1$ Universided de los Andes
\\
$^2$ a
\normalsize
\\
\\
\textbf{The Lyman-alpha emission line is a strong indicator of star formation.
Observations of this line are central to construct samples of the most
distant star forming galaxies, which in turn are useful for studies in
cosmology and galaxy evolution.  
The intepretation of Ly-$\alpha$ observations has to go together with
computational models for the radiative transfer of Lyman-alpha photons.
Recent theoretical work suggests that galaxy rotation has a
measurable impact on the shape of the Lyman-alpha line. 
Here we report that lyman-alpha observations of a dwarf compact galaxy (TOL
1214-277) in the local universe, which had previously evaded theoretical 
interpretation, are naturally explained by rotational effects.
We constrain the rotational velocity, viewing angle and total
neutral hydrogen mass from the Lyman-alpha observations.
These values are in broad agreement with other observational
constraints, although  they point towards a rather atypical physical
nature for the source.  
Our results present a new observational method to estimate the rotational
velocity of dwarf galaxies.
Considering the expected similarities between the local and the most
distant dwarf galaxies, we anticipate that the Lyman-alpha line could be
used to constraint the rotational state of the neutral gas in the
highest redshift galaxies to be detected with next generation infrared
spectroscopic facilities such as the James Webb Space Telescope.}



%\section{Introduction}

1. General paragraph about the Lyman alpha line.

2. General paragraph about modelling the Lyman alpha
line. Outflows.

3. Rotation and the expected features.
It has been shown that rotation also imprints an effect on the
Lyman-alpha morphology. 
The most important consequence of rotation is that spherical 
symmetry is broken.
The line morphology now depends on the viewing angle respect to the
rotation axis.  
For a line sight perpendicular to the rotation axis the intensity and
the line center and the line width increase with rotational velocity. 
When the rotational velocity is close to the half-line width of the
static line, the line becomes single peaked as it is observed in
\tol, a unique feature that other theoretical models find
impossible to reproduce.



%\section{TOL1214-277}

4. The charachteristics of the dwarf galaxy of interest.

TOL1214-277 was first observed by ... it is a compact 
dwarf galaxy and does not have old stars. 

The Ly$\alpha$ emission line was first observed in the TOL1214-277 
galaxy by \citep{Thuan97}. It has two main important features which 
make this a very uncommun LAE. First it shown a symmetric profile 
which is , Second the Ly$\alpha$ line is not shifted with respect to 
the H$\beta$ line.  Blue compact Dwarf Galaxy

	%\section{Results}

5. The results of the fit. 

Figure 1. Observed line + fit. 


%6. Comparison of these results against other observational data.

We estimate the total neutral hydrogen mass to be $M\approx
m_H\tau^3\sigma^{-3}n^{-2}$, where $m_H$ is the mass a Hydrogen's atom, $\tau$ is
the optical depth and $n$ is the number density of neutral Hydrogen
atoms. For this system we estimate that for average values of
$n=1\times 10^3$ the total hydrogen mass is $M\times
10^{14}$M$_{\odot}$. However, blind HI surveys have put an upper limit
in the neutral hydrogen mass of ???



7. Implications for outflow+rotation in existing samples.

8. Implications for very high-z dwarf galaxies. 

\end{document}

