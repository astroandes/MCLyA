\documentclass{article}
\title{Measurement of a dwarf galaxy rotational velocity from its Lyman-alpha
  line emission}
\author{MCRG, JNGC, JEFR}
\begin{document}
\maketitle
\begin{abstract}
The Lyman-alpha emission line is a strong indicator of star formation.
Observations of this line are central to construct samples of the most
distant star forming galaxies, which in turn are useful for studies in
cosmology and galaxy evolution.  
The intepretation of Ly-$\alpha$ observations has to go together with
computational models for the radiative transfer of Lyman-alpha photons.
Recent theoretical work suggested that galaxy rotation should have a
measurable impact on the shape of the Lyman-alpha line. 
Here we show that galaxy rotation can be measured from the Lyman-alpha
line profile. 
We found that lyman-alpha observations of a dwarf galaxy (Tol 1214-277) in
the local universe, which had previously evaded theoretical 
interpretation, are naturally explained by rotational effects.
We constrain the rotational velocity, viewing angle and total
neutral hydrogen mass from the Lyman-alpha observations.
These alues in agreement with other observational constraints.
Our results present a new observational method by which the rotational
velocity of a dwarf galaxy can be estimated. 
Considering the expected similarities between the local and the most
distant dwarf galaxies, we anticipate that the Lyman-alpha line could be
used to constraint the kinematic state of the highest redshift
galaxies to be detected with next generation infrared spectroscopic
facilities such as the James Webb Space Telescope.


\end{abstract}


\end{document}

